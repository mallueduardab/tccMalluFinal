\chapter{REFERENCIAL TEÓRICO}
\label{cap:referencial}

A Revolução da informação, tida como a terceira das maiores revoluções, é resultado da acelerada difusão das tecnologias da informação e comunicação (TICs). Com a modernização dos computadores e a instauração das Redes de Computadores, possibilitou o compartilhamento de recursos físicos e lógicos (\textit{e-maisl}, impressoras, entre outros) expandindo ainda mais a conexão entre tecnologia e pessoas \cite{helmano1995}.


O uso de computadores nas escolas, traduz o que pode-se chamar de pensamento crítico \cite{valente1993}. É através dele que se explica porque a utilização de tecnologia educacional transmite informações e corroba em serviços úteis à educação e a prática pedagógica, como por exemplo, promover o aprendizado de matemática por meio de jogos educacionais interativos em sala de aula \cite{seabra1993}.

Devido a essa modernização tecnológica, é notório que as organizações visam constantemente o alcance de processos de mudanças e inovações, em busca de qualidade e competitividade. Segundo Fonseca (2001), para alcançar esses objetivos, as empresas devem se preocupar com o envolvimento de seus colaboradores no processo produtivo/serviço, e consequentemente a qualidade de vida dos mesmos, o que envolve condições de trabalho, novas formas de gestão, saúde física e mental, entre outras. No entanto, ao analisar o sistema educacional, percebemos que o método de trabalho do professor ainda é estagnado, dificultando a efetividade e organização das funções cabíveis a eles.


Associar a tecnologia às práticas pedagógicas é um desafio a ser enfrentado: de um lado estão as ferramentas educacionais que tendem a facilitar as tarefas diárias e moderniza-las ao mesmo tempo, do outro, os professores, que alegam carecer de conhecimento frente a essas tecnologias \cite{REP's1158}. Logo, um dos meios de solucionar o problema é aderindo a substituição de ferramentas ultrapassadas utilizadas no processo educacional, a fim de facilitar o trabalho desses profissionais.

Como exemplo temos o Diário de Classe, descrito segundo o manual de orientação \cite{manualDiario} , “um instrumento de registro do planejamento e do desenvolvimento das atividades pedagógicas do(a) professor(a), instrumento legal de registro das situações didáticas da vida escolar dos(as) estudantes, do acompanhamento das suas aprendizagens e do desempenho escolar. No diário de classe devem constar: a relação nominal dos(as) estudantes, em ordem alfabética, observações sobre o rendimento, frequência justificada e atitudes comportamentais; o planejamento das aulas, o registro dos conteúdos trabalhados em situação didática de cada bimestre e as atividades ou projetos especiais.” . O diário é disponibilizado em forma de livro, onde o preenchimento deve ser feito manualmente e sem rasuras, sendo elas rubricadas e justificadas quando acontecido \cite{teresa1999manual}.


Nas escolas, exitem vários relatos da utilização de TICs, bem como sua viabilização e aceitação. Kasin e Silva (2008) relata como o computador e a \textit{Internet} tornam-se aliados no processo ensino-aprendizagem, integrando em sua metodologia softwares autorais para criação coletiva, quando o professor busca conhecimentos amplos na utilização de tecnologias educacionais, mediando todo o processo. Outro exemplo, é a utilização do Diário Escolar Digital, uma "plataforma idealizada pela Secretaria de Estado de Educação de MInas Gerais - SEEMG - e desenvolvida pela Prodemge, cuja proposta é ampliar a interação entre estudantes, pais, responsáveis e profissionais da educação (Professores, Especialistas, Diretores e Secretários de Escola)" segundo o site Escola Interativa.
 

Elas devem agir de forma a complementar o ensino tradicional, e jamais substituir a instituição principal; a escola. O professor por sua vez, deve incrementar em seu currículo e, simultaneamente em suas aulas, uma formação ampla para permear a difusão de conhecimento através da utilização dessas tecnologias  \cite{albino2016avaliaccao}. Sendo assim, tecnologias voltadas à educação tendem a complementar e modificar a prática pedagógica afim de surtirem  facilidades e modernização de ferramentas e métodos. Entretanto, é importante ressaltar que o professor é ferramenta primordial nesse processo.


Ponte (2002), afirma que " Na escola, as TIC são um elemento constituinte do ambiente de aprendizagem. Ou seja, é parte do sistema para acarretar benefícios. "[...]Representam, além disso, uma ferramenta de trabalho do professor e do educador de infância e um elemento integrante da sua cultura
profissional" \cite[p. 2]{ponte2002tic}. 


