\chapter{INTRODUÇÃO}
\label{cap:introducao}


A constante busca pela facilidade de acesso à informação, tem motivado cada vez mais o desenvolvimento de sistemas Web. Estes por sua vez, diferentemente do modelo tradicional, não precisam ser instalados e são acessíveis de qualquer lugar, por qualquer dispositivo conectado à internet. Atualmente, esse tipo de tecnologia é voltada para as mais diversas áreas e é empregada com várias finalidades, inclusive na educação.

No campo educacional, a gestão escolar retoma os conceitos de eficiência e melhoria no sistema de ensino, e é por meio dela que se busca organizar as áreas da escola. Nesse contexto, a adoção de sistemas para auxiliar a gestão escolar, automatizando o processo, traz benefícios como o aumento na produtividade dos funcionários, dados concretos e facilidade de acesso à informação. 


Muitas escolas da atual rede de ensino brasileira, ainda adotam a utilização do Diário de papel pelos professores. Esse, é o documento oficial no qual são registradas todas as atividades pedagógicas abordadas, bem como o histórico de presença e notas de cada aluno, e é posteriormente repassado à secretaria ao final de cada período letivo, através da entrega da taleta. Tal processo é manual, lento e passível de erros, o que caracteriza a necessidade da adoção de um sistema mais eficiente e seguro a fim de melhorar o processo de gestão dessas informações.



Tomando como ponto de partida o estudo "Diários de Classe: traços históricos de um ensino de língua" de Menegolo e Cardoso, que analisou a aplicação real desse material, concluiu-se que as anotações realizadas pelos professores não condizem com as aulas lecionadas, mas sim com uma cultura escolar predeterminada que visa suprir as cobranças de um sistema de ensino frágil. Tal omissão corrobora para dificultar um diagnóstico preciso e correto sobre as falhas educacionais e as áreas que carecem de mudança imediata. Nesse sentido, a utilização de ferramentas tecnológicas, como sistemas Web, tem sido modelados para auxiliar na busca de uma solução prática e segura na tentativa de amenizar tais problemas.



A busca por ferramentas para auxiliar no problema abordado, levou a criação de projetos como o Educação Conectada\footnote{http://www.educacaoconectada.com.br/}, que desenvolveu um aplicativo móvel para dispositivos Android\footnote{https://www.android.com/} que desempenha a mesma função do diário de papel supracitado. Entretanto foram identificados problemas na interface da aplicação, quanto as funcionalidades, e foram tratadas para melhor atender as necessidades do usuário final segundo padrões e técnicas de \textit{design} centrado no usuário, produzindo como resultado final, um novo protótipo de interface \cite{igor2016}.



\section{Objetivos}


\subsection{Objetivo geral}

Este trabalho tem como objetivo geral desenvolver um sistema Web para o perfil do professor, que funcione como um Diário de Classe virtual, substituindo o Diário de papel. A interface da aplicação será baseada nos resultados obtidos por Lacerdino e Melchiori (2016).


\subsection{Objetivos Específicos}


Os objetivos específicos norteiam: 

\begin{itemize}
	\item Estudo das ferramentas necessárias para o desenvolvimento da aplicação;
	\item Especificação de requisitos: levantamento de requisitos, diagrama de casos de uso e protótipo de interface;
	\item Projeto do software: definição dos módulos e componentes, criação do banco de dados e padrão de arquitetura MVC;
	\item Implementação: codificação;
	\item Teste: dinâmico, com validação das funcionalidades através da interface (teste de caixa branca) durante a implementação;
	
\end{itemize}












