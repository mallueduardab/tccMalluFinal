\chapter{REFERENCIAL TEÓRICO}
\label{cap:referencial}


No desenvolvimento de \textit{softwares} Web, a principal busca é a satisfação das necessidades dos usuários. A alta demanda por esse tipo de serviço é justificada pelo avanço das tecnologias de gestão da informação, e as facilidades que a utilização da internet trouxe em relação ao acesso e disponibilização de informação, praticidade e segurança \cite{barbosa2009gestao}. Assim, o emprego dessa ferramenta está nas mais diversas áreas e com amplas propostas de utilização .

As facilidades alcançada com a utilização de \textit{softwares} Web, vão além diversidade de ferramentas tecnológica que conseguem ter acesso a elas. Blatmann, Fachin e Varkavis (2000) ressalta que  "O valor dos hiperdocumentos não está nos custos de manutenção mas na forma do usuário aprender mais rápido, localizar a informação mais rapidamente, ou acessar o conhecimento armazenado mais efetivamente.". Isso permite que qualquer tipo de usuário se beneficie de tal tecnologia.

Bax e Parreiras (2003) enfatizam os aspectos da descentralização da informação e os toma como justificativa para o desenvolvimento de sistemas de gestão de conteúdos na Web. Em um sistema onde o funcionamento gira em torno de vários módulos, melhorias são alcançada com a divisão das responsabilidades e a facilidade de acesso à elas, de acordo com sua função na hierarquia.

Atualmente, muito se discute acerca das atuais práticas de gestão da educação bem como metodologias que podem ser aplicadas e/ou inseridas no ambiente, para dinamizar os processos e promover a eficiência e democratização dessa prática. As principais problematizações norteiam a realidade educacional brasileira e a formação de profissionais \cite{paro2007gestao}.

Pensando em formas de intervenção no mundo da educação, o processo de gestão das organizações escolares requer etapas de profundas mudanças \cite{brito2006tecnologias}. Essas, permeiam profissionais, alunos e toda prática pedagógica adotada dentro do processo de adaptação às mudanças, bem como a escolha das ferramentas e sua forma de utilização para obtenção de retornos positivos.

Devido à modernização tecnológica as organizações tem buscado alcançar processos de mudanças e inovações, visando qualidade e competitividade. Segundo Fonseca (2001), para alcançar esses objetivos as empresas devem se preocupar com o envolvimento de seus colaboradores no processo produtivo/serviço, e consequentemente a qualidade de vida dos mesmos, o que envolve condições de trabalho, novas formas de gestão, saúde física e mental, entre outras. No entanto, ao analisar o sistema educacional, percebemos que o método de trabalho do professor ainda é estagnado, dificultando a efetividade e organização das funções cabíveis a eles.

Associar o uso de sistemas Web à gestão das práticas pedagógicas é um desafio a ser enfrentado: de um lado estão os sistemas desenvolvidos para facilitar as tarefas diárias e moderniza-las ao mesmo tempo; do outro os professores, que alegam carecer de conhecimento frente a essas tecnologias \cite{REP's1158}. Logo, um dos meios de solucionar o problema é aderir à substituição de ferramentas ultrapassadas utilizadas no processo educacional, a fim de facilitar o trabalho desses profissionais.


Essas ferramentas de auxílio, devem agir de forma a complementar o ensino tradicional, e jamais substituir a instituição principal; a escola. O professor por sua vez, deve incrementar em seu currículo e, simultaneamente em suas aulas, uma formação ampla para permear a difusão de conhecimento através da utilização dessas tecnologias  \cite{albino2016avaliaccao}. Sendo assim, tendem a complementar e modificar a prática pedagógica afim de surtirem  facilidades e modernização de ferramentas e métodos. Entretanto, é importante ressaltar que o professor é ferramenta primordial nesse processo e carece de ferramentas de gestão que facilite tal metodologia.


Como exemplo de parte das atribuições de responsabilidade do professor, temos o preenchimento do Diário de Classe. Descrito segundo o Manual de Orientação de Zaponi, Monteiro e Torres (2009) é “um instrumento de registro do planejamento e do desenvolvimento das atividades pedagógicas do(a) professor(a), instrumento legal de registro das situações didáticas da vida escolar dos(as) estudantes, do acompanhamento das suas aprendizagens e do desempenho escolar. No diário de classe devem constar: a relação nominal dos(as) estudantes, em ordem alfabética, observações sobre o rendimento, frequência justificada e atitudes comportamentais; o planejamento das aulas, o registro dos conteúdos trabalhados em situação didática de cada bimestre e as atividades ou projetos especiais.”.

O diário é disponibilizado em forma de livro, onde o preenchimento deve ser feito manualmente e sem rasuras, sendo elas rubricadas e justificadas quando acontecido \cite{teresa1999manual}. Como consequência, tal processo requer maior atenção, é passível de erros, repetitivo e cansativo. Nesse contexto, aplicações Web de boa qualidade podem ajudar e muito  

Igor e Melchiori (2016) discorrem sobre o impacto da substituição de antigos métodos de gestão, enfatizando também acerca da problematização que tange sua utilização. Ao fazer a análise de uma aplicação de gestão educacional desenvolvida no projeto Educação Conectada para substituição do Diário de Classe de papel nas escolas participantes, apontam falhas em relação a usabilidade da interface e conceitos técnicos que devem ser levados em consideração na criação de aplicações de qualidade.

A Engenharia de aplicações Web (\textit{Web Engineering}), utiliza de princípios da engenharia de software para desenvolver aplicações Web de qualidade, que se baseiam na corretude e completude  da aplicação, segundo as necessidades do usuário, para obtenção do produto final \cite{pressman2011engenharia}. Estas fases são: Especificação dos requisitos, Projeto, Implementação, Teste e Manutenção.

Dentro da engenharia de requisitos, o processo de especificação deve englobar a elicitação, modelo e análise dos requisitos levantados. Estes, tem por finalidade promover uma especificação clara, não ambígua e completa sobre as necessidades do cliente \cite{koscianski2007qualidade}.

A fase de projeto decide como a aplicação vai operar em termos de arquitetura, interface do usuário e base de dados. É através dessas definições que se estabelece e decide a viabilidade do projeto \cite{pressman2011engenharia}.

Modelagem e projeto do banco de dados, influenciam totalmente o comportamento e desenvolvimento da aplicação. O modelo relacional introduz as linguagens de consulta de alto nível como SQL (\textit{Structured Query Language}), trazendo facilidades para programadores, é disponibilizado pela maior parte dos SGBDs (Sistemas Gerenciadores de banco de dados) e utiliza representação em tabelas \cite{elmasri2005sistemas}.

A maioria dos navegadores oferece suporte ao novo HTML5. A linguagem de marcação, combinada com os \textit{styles} possibilitado pelo uso de CSS, agregam valor a aplicação desde o \textit{design} à usabilidade, pelo usuário. \textit{Framewokrs} como o Bootstrap deixam a aplicação responsiva e o \textit{layout} comportado em qualquer tipo e tamanho de tela, revolucionando os antigos métodos \cite{kim2013responsive}.

Já na implementação, tem-se o código fonte da aplicação. Cabe ao programador, através do estudo de viabilização e disponibilidade das ferramentas, a escolha das tecnologias utilizadas, linguagem de programação e padrões de projeto. Nesse momento, deve ser levado em consideração questões como desempenho, usabilidade  e tendencias de mercado \cite{dantas2002suporte}.

Como linguagem de programação para Web, PHP se destaca pela vasta interação na dinamização de páginas estáticas, ser gratuito, orientado a objetos, embutido no html, executado no servidor, suportar vários bancos de dados e ser altamente portável para execução em Windows, Linux ou Unix \cite{niederauer2004desenvolvendo}.


Segundo Dantas (2002) "Podemos pensar em um padrão como a reutilização da essência de uma solução para determinados problemas similares.". Na construção de aplicações Web, a demanda pela utilização de arquiteturas de \textit{softwares} que seguem um padrão crescem cada vez mais, pois facilita a legibilidade, manutenção e compreensão do código. O padrão MVC (\textit{Model-view-controller}) propõe a separação da representação, da informação da sua apresentação \cite{cavalcanti2006arquitetura}. 


Contudo, além de disponibilizar uma vasta gama de áreas de utilização, softwares Web são desenvolvidos com base em fatores de desempenho, boas práticas de programação e padrões estruturais, para melhor alcançar os resultados propostos. Como ferramenta de auxilo à gerencia escolar, tende a diminuir o acúmulo de responsabilidades com a separação em módulos, facilidade de acesso à informação, agilidade nos processos e maior segurança e integralidade dos dados \cite{barbosa2004contribuiccao}.


